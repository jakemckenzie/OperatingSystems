\documentclass[12pt]{article}
\usepackage[shortlabels]{enumitem}
\begin{document}
\title{TCSS 343 - Week 4}
\author{Jake McKenzie}
\maketitle
\noindent\centerline{\textbf{Homework 1}}
1.4 Keeping in mind the various definitions of operating system, consider
whether the operating system should include applications such as web
browsers and mail programs. Argue both that it should and that it
should not, and support your answers.\\\\

Answer: My argument for an operating system with an email client and web browser built in 
would be that larger more involved environments benefit in some functional 
way by including such features. Windows and Ubuntu, for example, are served by 
including such applications because they are desktop environments.  \\\\

My argument for an operating system without an email client 
and web browser built in would be for smaller environments 
typically not used on the desktop. These would include smaller 
embedded devices such as printers and LeapFrog computers for kids. 
These embedded devices are smaller and have more focused devices 
for specific tasks.\\\\
1.6 Which of the following instructions should be privileged?\\\\
\begin{enumerate}[a)]
    \item Set value of timer. ANSWER: \textbf{Privileged} From page 26 and 27. 
    Before turning over control to the user, the operating system ensures 
    that the timer is set to interrupt. If the timer interrupts, control 
    transfers automatically to the operating system, which may treat the 
    interrupt as a fatal error or may give the program more time. 
    Clearly, instructions that modify the content of the timer are 
    privileged.
    \item Read the clock. ANSWER: \textbf{Unprivileged} Not mentioned in the
    text but it should be obvious that every process should be able to read 
    the clock because otherwise every process would be a mess and no process 
    could get anything done. 
    \item Clear memory. ANSWER: \textbf{Privileged} Not mentioned in the 
    text but it should be obvious that memory is sacred and requires 
    privileged access. If a process does not have access to a piece of 
    memory then the OS needs to deny access to that process.
    \item Issue a trap instruction. \textbf{Unprivileged} I can't find 
    this in the text but it would seem obvious to me from lecture that 
    all processes need some way to terminate themselves.
    \item Turn off interrupts. \textbf{Privieleged} From 
    page 52. Various instructions are privileged and can be executed only 
    in kernel mode. Examples include the instruction to switch to kernel 
    mode, I/O control, timer management, and interrupt management.
    \item Modify entries in device-status table. \textbf{Privileged} From 
    page 52. Various instructions are privileged and can be executed only 
    in kernel mode. Examples include the instruction to switch to kernel 
    mode, I/O control, timer management, and interrupt management.
    \item Switch from user to kernel mode. \textbf{Privileged} From page 52.
    Various instructions are privileged and can be executed only in kernel
    mode. Examples include the instruction to switch to kernel mode, I/O
    control, timer management, and interrupt management. 
    \item Access I/O device. \textbf{Privileged} From page 52. Various 
    instructions are privileged and can be executed only in kernel mode. 
    Examples include the instruction to switch to kernel mode, I/O
    control, timer management, and interrupt management.
\end{enumerate}
1.8 Some CPUs provide for more than two modes of operation. What are two
possible uses of these multiple modes?\\\\
Answer: Supervisor mode and privileged mode, both of which allow for more varied 
types of access through a varying hierarchy protection of domains.\\\\
1.12 In a multiprogramming and time-sharing environment, several users
share the system simultaneously. This situation can result in various
security problems.\\\\
\begin{enumerate}[a)]
    \item What are two such problems? Identity theft and theft of service 
    often security problems that arise in multiprogramming and time-sharing 
    engironments. 
    \item Can we ensure the same degree of security in a time-shared 
    machine as in a dedicated machine? Explain your answer. We cannot 
    ensure the same degree of security in a time-shared environment than 
    dedicated enviornment because cannot lock down that system in the same
    way. By it's nature a time-shared machine is more complicated than 
    a dedicated machine. More complexity means a higher propensity for 
    the possibility of bugs and less likely that resources will be allocated 
    fairly. This is not a given that this will be the case but it is the obvious 
    answer to me.
\end{enumerate}
1.14 Under what circumstances would a user be better off using a timesharing
system than a PC or a single-user workstation?\\\\
An Amazon S3 server with multiple virtual machines is an instance of a 
time sharing system. Often times the economics of renting time on a 
large machine is better than owning your own hardware.\\\\
1.19 What is the purpose of interrupts? \\\\
Answer: Assuming the question means hardware interrupt, this is a signal raised 
by a process or a external piece of hardware to disturb a process running on 
piece of hardware. Interrupts are asynchronous and may occur at any time.\\\\
How does an interrupt differ from a trap? \\\\
Traps are raised by user programs to invoke some functionality in the OS.\\\\
Can traps be generated intentionally by a user program? If so, for what 
purpose?\\\\
Answer: That's the whole point of a trap is to invoke some functionality of an 
operating system or catch arithmetic errors(depending on the hardware). 
So yes, a trap can be generated intentionally by a user program.\\\\
1.25 Describe a mechanism for enforcing memory protection in order to
prevent a program from modifying the memory associated with other
programs.\\\\
Answer: Virtual memory is one of the most used mechanisms for enforcing memory 
protection in order to prevent a program from modifying the memory associated 
with other programs. \\\\
1.26 Which network configuration—LAN or WAN—would best suit the
following environments?
\begin{enumerate}[a)]
    \item A campus student union Answer: LAN because student unions are held 
    in one building.
    \item Several campus locations across a statewide university system Answer: 
    WAN because the distance between computers in the network are vast.
    \item A neighborhood Answer: WAN because the distances between houses 
    can be greater than 300 feet which is about the limit of the distance 
    for most ethernet cables.
\end{enumerate}
1.27 Describe some of the challenges of designing operating systems for
mobile devices compared with designing operating systems for traditional
PCs.\\\\
Answer: Power consumption is the most obvious challenge for mobile devices because 
batteries are a deficient technology. Most desktop PCs assume they have 
access to infinite power, mobile devices cannot make these assumptions. 
Most medical environment don't talk on even local networks inorder to 
lockdown the system from outside attack, while traiditional PCs see this
as a feature most of the time. Operating systems on traditional PCs had at 
one time a greater expection of features than mobile operating systems,
but has changed with the advent of the smart phone but is true (mostly).
\end{document}