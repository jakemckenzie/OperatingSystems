\documentclass[12pt]{article}
\usepackage[shortlabels]{enumitem}
\begin{document}
\title{TCSS 343 - Week 4}
\author{Jake McKenzie}
\maketitle
\noindent\centerline{\textbf{Homework 2}}
2.10 Why do some systems store the operating system in firmware, while
others store it on disk?\\\\
\textbf{ANSWER: }Embedded devices are typically smaller, many would
like to have to map ROM directly to memory. As such they have no need 
for a file system or any of niceties that go along with systems that 
store the OS on disk. The gameboy is an example of such an embedded 
device. This gameboy would map ROM directly to memory, storing the 
OS in firmware. To change the OS on the gameboy you would need to flash 
the firmware. 
2.16 What are the advantages and disadvantages of using the same 
systemcall interface for manipulating both files and devices?\\\\
\textbf{ANSWER: }A major advantage of interoperability between 
using the same systemcall interface for manipulating both 
files and devices is allowing for a higher level of abstraction.
This abstraction leads to less code re-use. A major disadvantage 
of this is the potential to lose performance, but this is a 
disadvantage that is harder to pin down. Whenever you add any 
layer of abstraction to a system you include the potnetial for 
the loss of functionality or performance. That said, I think 
the advantage of say, UNIX's ability to treat hardware devices 
as files is a major advantage outweighing most disadvantages.
2.18 What are the two models of interprocess communication? What 
are the strengths and weaknesses of the two approaches?\\\\
\textbf{ANSWER: }The two models of interprocess communication 
are the message-passing model and the shared-memory model. 
Since message-passing is a system call, it has a slightly 
higher overhead than shared-memory model. This lack of system 
calls on the other hand mean than it takes more overhead from
the programmer to take care of how the memory is shared.
2.19 Why is the separation of mechanism and policy desirable?
\end{document}